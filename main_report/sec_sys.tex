\section{СТРУКТУРНОЕ ПРОЕКТИРОВАНИЕ}
\label{sec:sys}

В данном разделе описана структура локальной компьютерной. Разработка структурной схемы требуется 
для упрощения разработки локальной компьютерной сети. Структурная схема позволяет разделить
разрабатываемую локальную компьютерную сеть на логические блоки, и позволяет иметь представление о разрабатываемой 
локальной компьютерной не вникая в детали, и реализацию программно-аппаратных средств. Локальную компьютерную сеть 
здания следует разделить на следующие блоки:

\begin{itemize}
    \item блок \blockRoute
    \item блок \blockSwith
    \item блок \blockDevices
    \item блок \blockVideo
    \item блок \blockAccessPoint
\end{itemize}

Всего предполагается организовать 50 стационарных подключений. В мобильных подключениях заказчик не уверен, 
но их предполагается реализовать в силу того, что сейчас большинство мобильных устройств взаимодействуют с сетью благодаря 
беспроводному подключению. Помимо вышеперечисленных подключений в здании к локальной сети будут подключены принтеры и камеры видеонаблюдения.
взаимодействие всех структурных блоков показано на схеме \structScheme.

\subsection{Блок \blockRoute}   

Блок маршрутизации является ключевым элементом инфраструктуры локальной компьютерной сети организации. Он представляет собой важное звено, 
обеспечивающее бесперебойное функционирование всей сети. Этот блок ответственен за маршрутизацию данных внутри организации и обеспечивает 
доступ к Интернету для всех подключенных устройств.
С одной стороны, маршрутизаторы этого блока подключены к провайдеру интернет-услуг, что позволяет организации получать доступ к глобальной сети. 
С другой стороны, они надежно связаны с блоком коммутации, что обеспечивает передачу данных между всеми устройствами в локальной сети.
Используя технологию DOCSIS (Data Over Cable Service Interface Specification), маршрутизаторы эффективно управляют потоком данных, обеспечивая 
высокую скорость и надежность соединения. Таким образом, блок маршрутизации является сущностным элементом сети, обеспечивая ее связность, 
безопасность и доступность, что позволяет всем устройствам в организации успешно взаимодействовать между собой и с внешним миром.

\subsection{Блок \blockSwith}   

Блок \blockSwith\ представляет собой важное звено в структуре локальной компьютерной сети организации. 
Блок взаимодействует практически со всеми остальными блоками в локальной сети организации. 
Он состоит из нескольких коммутаторов, объединенных в иерархию, которая обеспечивает эффективное управление и передачу данных внутри сети. 
Общее взаимодействие блока коммутации с другими блоками включает в себя передачу данных от маршрутизаторов через коммутаторы к конечным 
устройствам, а также обеспечение беспроводного доступа к сети через точки доступа. 

Первым из блоков с которым взаимодействует блок \blockSwith\ является блок оконечных устройств. 
Каждый коммутатор в этом блоке подключен к группе стационарных ПК, представленных блоком \blockDevices\@. 
Коммутаторы обеспечивают связность и коммутацию данных между этими ПК внутри организации.

Следующий блок взаимодействующий с блоком \blockSwith\@, является блок \blockAccessPoint\@. 
Данный блок предоставляет беспроводное подключение для устройств, таких как ноутбуки и мобильные устройства. 
Коммутаторы взаимодействуют с точками доступа, обеспечивая беспроводную связь для сотрудников,
что позволяет им работать в сети без физического подключения к коммутаторам.

Последний блок связанный с блоком коммутации является блок \blockVideo\@. 
Коммутаторы в этом блоке обеспечивают передачу видеопотока. 
Это позволяет вам контролировать и записывать видеонаблюдение в разных участках организации.

\subsection{Блок \blockDevices}   

Следует определить, что из себя представляет блок \blockDevices\@. 
В локальной компьютерной сети организации представляет собой совокупность компьютеров (далее ПК), 
которые подключены к сети через коммутаторы. А также принтеры которые подключены к ПК.
Эти оконечные устройства выполняют разные функции в сети и взаимодействуют друг с другом, а также с другими блоками в сети. 
ПК в этом блоке представляют собой рабочие станции для сотрудников организации. Они могут использоваться для выполнения различных задач, 
таких как создание и редактирование документов, отправка и прием электронной почты, работа с приложениями и доступ к Интернету.
Взаимодействие между ПК и другими устройствами включает в себя отправку и прием данных через сеть.
Принтеры в этом блоке используются для печати документов и материалов, созданных на ПК сотрудников. 
Взаимодействие ПК с принтерами позволяет пользователям отправлять задания на печать, контролировать печать, а также получать готовые документы.

Ниже рассмотрим взаимодействие внутри данного блока и снаружи:
\begin{enumerate}
    \item Внутри блока ПК взаимодействуют между собой напрямую через коммутаторы. 
    Они могут обмениваться данными, создавать общие сетевые ресурсы, такие как общие папки, и работать в коллективе над проектами и заданиями.
    
    \item Снаружи ПК в блоке оконечных устройств могут взаимодействовать с другими блоками, 
    такими как блок маршрутизации и блок беспроводных устройств, через коммутаторы и сеть. 
    Это означает, что они могут отправлять запросы на доступ в Интернет через блок маршрутизации и,
    если есть необходимость, подключаться к беспроводным сетям через точки доступа, если они поддерживают Wi-Fi соединение.
    А также просматривать видеопоток от блока видеонаблюдения, если у конкретного ПК имеются для этого необходимые права. 
\end{enumerate}

\subsection{Блок \blockVideo}

Блок \blockVideo\ в локальной компьютерной сети организации состоит из камер видеонаблюдения, распределенных по зданию, и NVR (Network Video Recorder), 
который записывает и хранит видеопотоки от этих камер. 
Общая структура блока видеонаблюдения позволяет организации следить за безопасностью и наблюдать за событиями в разных участках здания. 

NVR подключен к коммутатору. Это соединение позволяет NVR получать видеоданные от камер и управлять ими. 
Также, NVR может быть удаленно управляемым, что позволяет вам просматривать видеозаписи и контролировать камеры через сеть.
Кроме того, NVR обычно обладает большим объемом хранилища для длительной записи видео, 
и может быть настроен для хранения записей в течение долгих периодов времени.

Соединение NVR с коммутатором обеспечивает передачу видеопотоков с камер видеонаблюдения в NVR.
Это также позволяет администраторам и персоналу сети получать доступ к записям и управлять видеонаблюдением из центральной точки.
Видеоданные, передаваемые через коммутатор, могут быть защищены с использованием сетевых технологий для обеспечения безопасности видеопотоков.

\subsection{Блок \blockAccessPoint}

Блок беспроводной сети в организации представляет собой систему беспроводной связи, 
которая обеспечивает беспроводный доступ к сети для устройств, поддерживающих Wi-Fi.

Основными компонентами блока беспроводной сети являются точки доступа (Access Points). 
Точки доступа ~-- это устройства, которые создают беспроводные сети Wi-Fi и предоставляют возможность устройствам подключаться к сети без использования проводов.
Точки доступа устанавливаются в разных местах здания таким образом, чтобы они обеспечивали равномерное покрытие Wi-Fi внутри помещения.
Они подключены к сетевому коммутатору и обеспечивают связь между беспроводными устройствами и проводной локальной сетью.

Блок беспроводной сети предоставляет беспроводной доступ к сети для различных беспроводных устройств, 
таких как ноутбуки, смартфоны, планшеты и другие устройства с поддержкой Wi-Fi. 
Эти устройства могут подключаться к беспроводной сети, используя точки доступа, и получать доступ к ресурсам сети, 
включая Интернет и совместные файлы, что позволит сотрудникам организации пользоваться беспроводным оборудованием.

Блок беспроводной сети может интегрироваться с другими блоками сети, включая блок \blockDevices и блок \blockVideo\@. 
Это позволяет беспроводным устройствам получать доступ к ресурсам в локальной сети. Доступ в Интернет обеспечивает связь между блоком \blockSwith\@.