\section{ОБЗОР ЛИТЕРАТУРЫ}
\label{sec:domain}

\subsection{DOCSIS}

DOCSIS (Data Over Cable Service Interface Specifications) ~-- обозначается стандарт передачи данных по телевизионному кабелю. 
Наиболее современная версия стандарта — DOCSIS 3.1. Отличается от предыдущих версий (1.0 ~-- 3.0) большей скоростью передачи
входящего и исходящего канала, а также саособом модуляции сигнала. Также имеется версия стандарта,
называемая EuroDOCSIS. Интерфейс между модемом и компьютером довольно традиционен. Обычно это
Ethernet или, иногда, USB. 

На стороне оператора должна находится распределительная станция, которая 
высокосортным оптоволоконным интерфейсом подключена к провайдеру. Это 
распределительное устройство называют CMTS (Cable Modem Termination System —
Оконечное устройство кабельного модема). 

При включении кабельного модема он начинает прослушивать входящий канал
в поисках специального пакета, время от времени посылаемого распределителем,
­чтобы получить системные параметры для модемов, только что включившихся
в ра­боту. После обнаружения данного пакета новый модем объявляет о своем по-
явлении по одному из исходящих каналов. Распределитель отвечает, присваивая
модему входящий и исходящий канал. Это распределение каналов может быть динамически 
изменено распределителем, если он решит, что необходимо сбалансировать нагрузку.

Остановимся подробнее на стандарте DOCSIS 3.1. Данный стандарт имеет производительность до 10 Гб/сек в прямом канале и до 2 Гб/сек в обратном. 
Это намного больше, чем у предыдущих стандартов и сравнимо со скоростями передачи данных по оптоволокну. Достигаются такие скорости 
за счет мультиплексирования с ортогональным частотным разделением каналов (OFDM) и кода с малой плотностью проверок на чётность.

DOCSIS 3.1 на текущий момент имеет реализации от множества компаний, таких как Motorola, Netgar, ARRIS, SURFboard.

\subsection{Защита от перепадов напряжения}

Перепады напряжения – это кратковременные
или импульсные изменения значения питающего напряжения в обе стороны (уменьшение и
повышение). Перепады напряжения могут иметь разную частоту, амплитуду,
продолжительность. Это зависит от причины возникновения скачка. Перепады напряжения
случаются регулярно. Скачки 5-10\% считаются незначительными, но их нельзя допускать
при работе с приборами, где важна точность получаемых измерений. Это, как правило,
медицинское и лабораторное оборудование. Вследствие незначительных скачков
напряжения управляющие микросхемы и микропроцессоры могут, к примеру,
переключаться в другой режим и выдавать различные сбои. Перепады 10-25\%, как правило,
уже существенны. Они очень пагубно влияют на технику и сокращают их срок жизни в 2-3
раза. Ещё более высокие скачки напряжения могут привести к выходу из строя как
отдельных частей электрооборудования, например, блоков питания, осветительных
электроприборов, сенсорных панелей, так и к полному выходу из строя
электрооборудования, вплоть до возникновения аварийных ситуаций и пожаров. 

Способами защиты от скачков напряжения являются:
\begin{enumerate}
    \item Стабилизаторы напряжения;
    \item Устройство защиты от импульсных перенапряжений (УЗИП);
    \item Источники бесперебойного питания (ИПБ);
    \item Сетевой фильтр;
\end{enumerate}

В качестве защиты от импульсных перенапряжений, возникающих при грозовых разрядах, 
коротких замыканиях подходит УЗИП. В случае небольших скачков
напряжений как это обычно и происходит УЗИП бесполезен для обеспечения защиты от перепадов 
напряжений. Стабилизаторы напряжения
регулируют входное напряжение и стараются максимально приблизить его фактические параметры 
к номинальным значениям. Это позволяет
повысить качество электроэнергии сети до уровня, удовлетворяющего требованиям даже самого 
чувствительного к характеристикам электропитания оборудования.
Сетевой фильтр обеспечивает фильтрацию и сглаживание приходящих из сети помех. При наличии 
в составе варистора он будет защищать и от экстремальных перенапряжений. 
Современный ИБП является эффективным средством защиты от сетевых скачков, отклонений и 
колебаний, как и стабилизатор напряжения.
Главным отличием ИБП приборов от всех вышерассмотренных является способность обеспечить 
бесперебойное питание нагрузки при отсутствии напряжения в основной сети. 

\subsection{Межсетевой экран Cisco IOS}



\subsection{Network video recorder}

NVR (Network Video Recorder) ~-- это устройство для записи и хранения видеопотока с IP-камер видеонаблюдения. 
Основное предназначение NVR ~-- это обработка, запись и хранение видеоданных с IP-камер для последующего просмотра или анализа.
NVR подключается к IP-камерам через локальную сеть, обычно по протоколу Ethernet или беспроводной сети Wi-Fi, 
и принимает видеопоток от каждой камеры. NVR обрабатывает и записывает видеопоток с каждой камеры на жесткий диск в реальном времени. 
Он сохраняет данные в виде файлов, обеспечивая хранение записей для последующего просмотра.
Пользователи могут использовать программное обеспечение NVR для просмотра записанных видеоматериалов, поиска определенных событий, 
анализа видеопотока и управления настройками системы видеонаблюдения.