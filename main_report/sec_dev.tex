\section{ПРОЕКТИРОВАНИЕ СТРУКТУРИРОВАННОЙ КАБЕЛЬНОЙ СИСТЕМЫ}
\label{sec:dev}

В данном разделе находится описание выбора кабелей, монтаж и размещение оборудования,
расчёт качества связи беспроводной сети для выстраиваемой ЛКС. Планом монтажа оборудования представлен в приложении В. 
Используемые условно-графические обозначения описаны в левой части схемы. Схема монтажная представлена в приложении В. 
Перечень оборудования, изделий и материалов представлен в приложении Д. 

В проектируемой ЛКС, прокладка кабельной подсистемы будет осуществляются вдоль стен на уровне ниже подоконников, в коробе 
Legrand Metra 638197
с поперечным сечением 80х80. 
Подключение к беспроводным точкам доступа Wi-Fi и камерам будет осуществлено пуском кабеля под потолком. 
Для всех подключений используется экранированная витая пара категории 6. 
Для рабочих станций используются информационные розетки, установленные в группах у рабочих мест. 
В иных случаях витая пара подведена напрямую к устройствам. 
В телекоммуникационном шкафу расположен модем, маршрутизатор, два коммутатора и NVR. Телекоммуникационный шкаф подвешен 
на высоте 2м от пола рядом с межэтажной шахтой в которой идут коаксиальные кабеля RG-11.