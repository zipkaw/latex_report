\section{ПРОЕКТИРОВАНИЕ СТРУКТУРИРОВАННОЙ КАБЕЛЬНОЙ СИСТЕМЫ}
\label{sec:dev}

В данном разделе находится описание выбора кабелей, розеток, монтаж и размещение оборудования,
расчёт качества связи беспроводной сети для выстраиваемой ЛКС. План этажа представлен в приложении В. 
Используемые условно-графические обозначения описаны в левой части схемы.
Перечень оборудования, изделий и материалов представлен в приложении Д. 

\subsection{Установка телекоммуникационного шкафа}

Для установки сетевого оборудования требуется изначально установить телекоммуникационный шкаф. 
К шкафу отсутствуют какие либо требования, в данной системе он требуется только для 
размещения оборудования в одном месте. В качестве решения был выбран шкаф "Cabeus WSC-4U" c размерами 225х326х300,
который подвешивается на высоте 2 метра от пола. 
в месте, близком к выходу коаксиального кабеля в помещение  организации.

\subsection{Установка коммутатора}

Коммутатор SW1 и SW2 следует закрепить в шкафу. Далее подключить кабеля в соответствии с функциональной схемой.
Затем следует подключить коммутаторы к сети.

\subsection{Установка модема}

Модем устанавливается в шкафу, к нему следует подключить коаксиальный кабель в соответствующий порт.
Кабель идущий к WAN-порту коммутатора следует подключить к Ethernet порту модема. 
После можно подключить блок питания, идущий в комплекте, к модему.

\subsection{Установка маршрутизатора}

Маршрутизатор крепится в шкафу, кабель идущий от модема подключается к WAN порту, согласно функциональной схеме.
К LAN-порту подключается коммутатор SW1 в порт согласно функциональной схеме. Затем следует подключить коммутаторы к сети.

\subsection{Установка NVR}

NVR крепится в шкафу, Ethernet-кабель от коммутатора SW-1 подключается к порту NVR согласно функциональной схеме.
Затем следует подключить коммутаторы к сети, блоком питания идущим в комплекте.

\subsection{Установка камер видеонаблюдения}

Камеры следует монтировать к потолку используя специальный крепеж идущий в комплекте, местоположение камер
обозначено на плане этажа. Так как камеры стоят в углу, а угол обзора камеры по горизонтали 101 градус, то следует ее повернуть так,
чтобы камера смотрела на 50 градусов от стены. Далее 
нужно провести по потолку кабель идущий от коммутатора SW1. Подключать питание к камерам не требуется.

\subsection{Установка точек доступа и расчет затухания сигнала}

Предполагается размещение точек доступа в центре комнаты 4 и у двери комнаты 2. Все точки доступа устанавливаются на потолке.
Для этого для их проведен кабель по потолку. Подключение питания для точек доступа не требуется, так как используется PoE.
Если учесть, что высота этажа три метра, то самая удаленная точка в любой из комнат находится на расстоянии 9-ти метров. 
Воспользуемся упрощенной формулой для расчета затухания сигнала:

\[L = 32,44 + 20lg(F) + 20lg(D), dB\],

Где \(F\) – это частота в ГГц, \(D\) – расстояние до самого удалённого мобильного устройства в метрах.

В качестве частоты возьмем 2,4 ГГц, так как эта частота является стандартной частотой. 
Рассчитаем затухание волны:

\[L = 32,44 + 20lg(2,4) + 20lg(9) = 59,13, dB\]

Мощность передатчика точки доступа составляет 20 дБм, для того чтобы узнать какой уровень сигнала будет на таком 
расстоянии следует от мощности передатчика отнять вычисленное выше затухание сигнала, итого уровень сигнала на 
таком расстоянии равен -39,13 дБм, что соответствует хорошему сигналу во всех местах помещений. 

Также, при проектировании плана этажа учитывалось то, что точки доступа будут использоваться в помещениях, где работают сотрудники
и при переходе из одного помещения в другое.

\subsection{Монтаж информационных розеток}

Для подключения стационарного оконечного оборудования (персональных пользовательских станций) около каждого рабочего места устанавливается 
по одной сдвоенной информационной розетке Valena Allure (Leg 753943) на высоте 40 см от пола. Для монтажа информационных розеток 
требуется выполнить следующие действия:

\begin{itemize}
    \item снять крышку путём отжатия наконечником отвёртки, вставленной в паз нижней или верхней стенки накладки;
    \item снять разъёмы путём выворачивания, отжав фиксирующие замки наконечником отвёртки со стороны, противоположной контактам;
    \item зачистить кабель, сняв внешнюю изоляцию на расстоянии 5 см и освободить провода;
    \item вставить провода с изоляцией (без зачистки) в контактные зажимы согласно маркировке и выполнить соединение при помощи фиксирующих колпачков;
    \item установить разъёмы, заведя под углом жестким фиксатором в соответствующие отверстия и защёлкнув фиксирующий замок.
\end{itemize}

Кабель к каждой розетке идет от коммутаторов и кабель выбирается согласно тому, где должна стоять та или иная рабочая станция.
Например, в помещении 2 размещается отдел бухгалтерии, управления и внедрения, поэтому для данной комнаты следует подключать кабеля к розеткам 
согласно функциональной схеме. Соотношение номеров розеток и портов коммутатора к каждому отделу приведено в таблице \ref{table:dev:rj45}. Номер розетки 
здесь обозначен в формате X1/X2/X3, где X1 ~-- номер помещения, X2 ~-- номер розетки, X3 ~-- номер RJ-45 разъема, для уточнения того,
какой разъем RJ-45 используется. На схеме обозначено всего лишь в формате X1/X2, так как там требуется уточнение местоположения розетки.

\begin{longtable}{
    | l
    | >{\raggedright\arraybackslash}m{0.200\textwidth}
    | >{\raggedright\arraybackslash}m{0.200\textwidth}
    | >{\raggedright\arraybackslash}m{0.230\textwidth}|}
    
    \caption{Соответствие розеток}
    \label{table:dev:rj45} \\
    \hline
    \centering\arraybackslash Отдел & 
    \centering\arraybackslash Номера розеток &
    \centering\arraybackslash Коммутатор &
    \centering\arraybackslash Номера портов коммутатора \\
    \hline
    \endfirsthead

    \caption{продолжение} \\
    \hline
    \centering\arraybackslash Отдел & 
    \centering\arraybackslash Номера розеток &
    \centering\arraybackslash Коммутатор &
    \centering\arraybackslash Номера портов коммутатора \\
    \hline
    \endhead

    Инженерии &
    4/19/1-4/25/1 &
    SW-2 &
    GE0/4, GE0/28-GE0/40
    \\
    \hline

    Разработки &
    4/9/1-4/14/1 &
    SW-2 &
    GE0/3, GE0/16-GE0/27
    \\
    \hline

    Тестирования &
    4/1/1-4/6/2 &
    SW-2 &
    GE0/2, GE0/5-GE0/15
    \\
    \hline

    Внедрения &
    3/2/1, 3/5/1, 3/8/1, 3/10/1 &
    SW-1 &
    GE0/4, GE0/11-GE0/13
    \\
    \hline

    Управления &
    2/6/1, 2/7/1, 2/10/1, 2/12/1 &
    SW-1 &
    GE0/3, GE0/8-GE0/10
    \\
    \hline

    Бухгалтерии &
    2/1/1, 2/3/1 &
    SW-1 &
    GE0/2, GE0/7
    \\
    \hline

    Видеонаблюдения &
    1/1/1 &
    SW-1 &
    GE0/5
    \\
    \hline

    Административный &
    1/1/2 &
    SW-1 &
    GE0/6
    \\
    \hline

\end{longtable}

\subsection{Прокладка кабелей}

К каждому рабочему месту необходимо провести кабель витой пары (UTP cat. 6) для подключения персональных пользовательских 
станций к локальной компьютерной сети и подключения к интернету. Все кабели стоит провести в кабель-канале для того, чтобы сотрудники 
не задевали кабели при передвижении по комнате. Для проведения проводов в кабель-канале был выбран Legrand
Metra 638197 с поперечным сечением 80х80.
Короб с кабелями требуется провести вдоль стен ниже уровня подоконника.
Для проведения кабелей к камерам и точкам доступа, следует проводить их за фальшпотолком. 

\subsection{Размещение оборудования}

Рабочие станции относящиеся к отделам управления, бухгалтерии и внедрения следует разместить в помещении 2, 
рабочие станции относящиеся к отделам разработки, инженерии и тестирования следует разместить в помещении 1, 
и подключить их к розеткам в порядке описанном в таблице \ref{table:dev:rj45}. К каждой станции из отдела следует 
подключить один принтер.

\subsection{Установка защиты от скачков напряжения}

Для каждой станции следует установить источник бесперебойного питания. К источнику бесперебойного питания подключить сетевой 
фильтр, монитор, а также сам ПК.
