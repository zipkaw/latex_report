\sectionCenteredToc{ВВЕДЕНИЕ}
\label{sec:intro}
\nocite{*}

В научно-исследовательской организации локальная компьютерная сеть необходима для организации рабочих моментов, проведения исследований, 
а также для обеспечения безопасности персонала. Интернет, системы безопасности и сервисы работают благодаря сети. Целью  курсового проекта 
является реализовать надежную и качественно сконструированную локальную компьютерную сеть для научно-исследовательской организации, 
которая занимается автоматизированными системами управления. Для выполнения поставленной цели требуется обозначить задачи, а именно: 
требуется изучить техническое задание, выбрать оборудование в соответствии с требованиями и обосновать данный выбор; описать настройку оборудования 
и составить руководство польз по проектированию локальной компьютерной сети; составить функциональную схему, план этажа и схему размещения 
оборудования, сделать соответствующие  выводы. Организация располагается в прямоугольном здании на первом этаже с площадью помещения в 450 метров 
квадратных и соотношением сторон 1 к 2. Известно, что в пределах офиса будет 50 пользовательских компьютеров и 100 стационарных подключений, 
в наличии мобильных подключений сети заказчик не уверен, но чтобы обеспечить комфортные условия для персонала, будет установлено беспроводное подключение к Интернету. 
Дополнительно заказчик потребовал задействовать уже имеющийся системный блок.
Из числа прочих оконечных устройств будут использованы по требованию заказчика принтеры и камеры видеонаблюдения.
Подключение к Интернет будет организовано через DOCSIS, внешний IPv4 адрес назначает провайдер, внутренний организован как приватная подсеть. 
IPv6 адресация будет предназначена для взаимодействия в рамках внутренней подсети. 
Для обеспечения безопасности сети заказчик выбрал сетевой экран, а для надежности локальной сети выбрал защиту от перепадов напряжения.
Производитель сетевого оборудования будет выбран из коммерческого сегмента сетевого оборудования. 

\nomenclaturex{DOCSIS}{Data Over Cable Service Interface Specifications}{Спецификация передачи данных через телевизионный кабель}

Разработанное решение было бы полезно не только для научно-исследовательской организации, а также для организаций, 
которые занимаются разработкой программного обеспечения, инженерно конструкторских организаций. Стоит отметить, что доступ в Интернет организован через
DOCSIS, использующий подключение через коаксиальный кабель, поэтому данный вариант будет подходящим для организаций, которые находятся далеко от города и 
где нет возможности проложить оптоволоконный кабель.
