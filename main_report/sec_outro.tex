\sectionCenteredToc{ЗАКЛЮЧЕНИЕ}
\label{sec:outro}

В ходе выполнения курсового проекта были применены на практике теоретические и практические знания, такие как
выбор оборудования, настройка и его размещение. Были выполнены все поставленные требования, а именно:

\begin{itemize}
  \item Учтена специфика объекта;
  \item Соблюдено количество стационарных пользователей и количество стационарных подключений;
  \item Соблюдено количество стационарных подключений;
  \item Установлены принтеры и видеонаблюдение;
  \item Организовано подключение к Интернету через DOCSIS;
  \item Адресация выполнена согласно заданию;
  \item Настроен сетевой экран;
  \item Установлена защита от перепадов напряжения;
  \item Задействован уже имеющийся системный блок;
\end{itemize}

Помимо требований задания, для организации качественной ЛКС была выполнена 
организация сети для мобильных подключений и было выбрано сетевое оборудование
компании Cisco. 

Среди достоинств данной сети можно выделить:

\begin{itemize}
  \item Защищенность сети;
  \item Безопасность персонала и имущества;
  \item Качественное оборудование и рабочие станции;
  \item Возможность расширения в будущем;
\end{itemize}

Недостатками построенной ЛКС являются: 

\begin{itemize}
  \item Высокая цена коммерческого оборудования;
  \item Использование DOCSIS для выхода в интернет;
\end{itemize}

В данной сети в будущем можно будет увеличить уровень безопасности
включением IPsec и VPN. Также в данную сеть можно будет добавить сервер
для хранения данных и web-сервер.