\sectionCenteredToc{ЗАКЛЮЧЕНИЕ}
\label{sec:outro}

В рамках данного дипломного проекта разработан и протестирован модуль приема и обработки GOOSE-пакетов в качестве части системы. Добавлена возможность протоколирования для упрощения дальнейшего использования.
Дипломный проект представлен в виде программного модуля для встраивания в систему любого проекта, который использует протоколы стандарта \iecStd. Разработанная система имеет следующие преимущества:

\begin{itemize}
    \item при получении поврежденного Ethernet-пакета он отбрасывается на этапе приема или обработки в зависимости от расположения поврежденных данных, что ускоряет работу системы;
    \item модуль имеет многопоточную архитектуру и способен одновременно обрабатывать несколько входящих пакетов.
\end{itemize}

К недостаткам и ограничениям дипломного проекта относятся:

\begin{itemize}
    \item отсутствие проведения модульного нагрузочного тестирования;
    \item при получении GOOSE-пакета с измененной структурой расположения параметров он отбрасывается, а полезные данные, которые имеются в этом пакете, не записываются в память.
\end{itemize}

Первый недостаток связан с сложностью создания некоторых функций-заглушек и эмуляции
работы других модулей.
Второй пункт является особенностью системы и обусловлен требованием описанного
поведения со стороны заказчика для упрощения понимания алгоритма работы полученного
устройства.

С учетом входных ограничений дипломный проект разработан в
полном объеме и качественно выполняет возложенные на него функции. В
дальнейшем он может быть улучшен следующими способами:

\begin{itemize}
    \item добавление возможности сохранения в память полезных данных некорректного пакета;
    \item добавление возможности получения и сохранения статистической информации о пришедших, обработанных и отброшенных GOOSE-пакетах в удобном для машинного взаимодействия виде;
    \item добавление возможности отправки GOOSE-пакетов;
    \item добавление поддержки других протоколов реального времени, описанных
    в стандарте \iecStd;
    \item добавление поддержки популярных архитектур и операционных систем,
    используемых во встраиваемых системах с поддержкой \iecStd.
\end{itemize}

Таким образом, разработанный модуль приема и обработки GOOSE-пакетов готов к дальнейшему внедрению в коммерческие проекты и при необходимости может быть адаптирован под нужды потребителя в сжатые сроки.
