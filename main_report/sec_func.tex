\section{ФУНКЦИОНАЛЬНОЕ ПРОЕКТИРОВАНИЕ}
\label{sec:func}

Данный раздел посвящён разработке функциональной схемы, выбору оборудования 
разрабатываемой локальной компьютерной сети, её функциональному проектированию и настройке. 
Схема функциональная приведена в приложении Б. 

\subsection{Выбор модели рабочей станции}

Выбор компьютеров для научно-исследовательской организации зависит от отделов организации. 
Например, для отдела инженерии, разработки и тестирования требуются станции способные запускать требовательное к ресурсам ПО, 
возможно с задействованием дискретной графики для моделирования процессов. 
В тоже время для отдела внедрения, управления и бухгалтерии не требуется техника, которая должна производить большое количество вычислений.
Прежде чем приступить к выбору конкретных компонентов, следует выделить состав любой пользовательской станции.

\begin{itemize}
    \item Операционная система;
    \item Процессор;
    \item Оперативная память;
    \item Жесткий диск;
    \item Графическая карта(встроенная либо дискретная);
    \item Сетевые адаптеры;
    \item Безопасность;
    \item Дисплей;
\end{itemize}

В соответствии с требованиями и составом пользовательской станции, которые приведены выше для отделов инженерии, разработки и тестирования
можно рассматривать станции с процессорами высокой производительности, например, Intel Core i7 или AMD Ryzen 7. Оперативной памяти должно быть 16 Гб 
или более для эффективной работы с большими объемами данных. В качестве жесткого диска лучшим решением будет SSD-накопитель для ускоренного процесса 
загрузки и работы с программами. Графическая карта будет выбрана исходя из того, что она будет необходима для качественного вывода изображения на мониторы,
также она должна соответствовать рекомендуемым требованиям ПО, которое будет использоваться внутри организации. На каждую пользовательскую станцию будет приходится
минимум 2 монитора, для комфортной работы персонала.

Для отдела внедрения управления и бухгалтерии требования для компонентов несколько ниже, в виду того, что персонал в данных отделах пользуется ПО предназначенным для 
работы с различными документами и ведения учета. Процессор можно использовать из серии Intel i3 или Ryzen 3. Количество оперативной 
памяти будет достаточно в количестве 8Гб. К жесткому диску и дисплею особых требований нет. Дискретная видеокарта не требуется, достаточно только встроенной в процессор.
В данный отдел также можно внедрить уже существующий системный блок !!!!!!!!!!!!!!!!!!!!!!!!!!!!!!!!!!!!!!!!!!!!!!!!!!!!!!!!!!!!!!!!!!!!!!!!!!!!

В качестве реальных решений будут рассмотрены готовые системные блоки, данное решение выбрано в целях экономии времени на монтажных работах и поиске комплектующих.
Выбор готовых решений будет производится на популярном сайте-агрегаторе товаров Onliner.by в которых есть возможность фильтрации товара, решение о товарах будет выбираться на основе рейтинга товара.

% Based on https://tex.stackexchange.com/a/352349/139966
\begin{table}[ht]
    \caption{Рабочие станции}
    \label{table:func:workStantionsProperty}
    \begin{tabular}{| >{\raggedright}m{0.310\textwidth}
                    | >{\raggedright\arraybackslash}m{0.310\textwidth}
                    | >{\raggedright\arraybackslash}m{0.310\textwidth}|}
        \hline
        \centering Название & \centering\arraybackslash Комплектующие & \centering\arraybackslash Средняя оценка/количество отзывов \\

        \hline
        FK BY GamePro 132719 &
        CPU AMD Ryzen 7 3700X 3600 МГц, RAM DDR4 16 ГБ, SSD+HDD 480 ГБ, графика: NVIDIA GeForce RTX 2060 Super 8 ГБ &
        4/10
        \\
        \hline
        FK BY RedGame 133164 &
        CPU Intel Core i7 9700K 3600 МГц, RAM DDR4 16 ГБ, SSD 480 ГБ, графика: AMD Radeon RX 5700 XT 8 ГБ &
        5/9
        \\
        \hline
        Компьютер Jet Wizard 7i8700K &
        CPU Intel Core i7 8700K 3700 МГц, RAM DDR4 16 ГБ, SSD+HDD 120 ГБ, графика: NVIDIA GeForce GTX 1050 Ti 4 ГБ &
        5/5
        \\
        
        \hline
    \end{tabular}
\end{table}

По количеству отзывов и максимальной средней оценке наиболее подходящий вариантом является станция FK BY RedGame 133164.

\subsection{Выбор операционной системы}

\subsection{Выбор модели принтера}

Принтер в НИО АСУ (НИО по автоматизированным системам управления) играет важную роль во всех отделах, 
обеспечивая печать документов, отчетов, графиков, схем, технической документации и других материалов.
Для принтера следует выделить следующие требования:

\begin{itemize}
    \item Качество печати и размер;
    \item Большой объем печати;
    \item Совместимость и подключение;
    \item Экономичность;
\end{itemize}

Качество печати принтера определяется количеством точек (DPI - dots per inch), которые принтер может разместить на дюйм. 
Чем выше DPI, тем более детализированное и четкое изображение будет напечатано. 
Большинство принтеров имеют разрешение от 600 до 4800 DPI. Также качество печати зависит от технологии печати: разные типы принтеров 
(лазерные, струйные, сублимационные и т. д.) используют разные технологии печати. Например, лазерные принтеры, обычно, обеспечивают 
более четкую и профессиональную черно-белую печать, в то время как струйные принтеры обладают лучшими возможностями для цветной печати.
Размер печати для данной организации подойдет от A4 до А3. А3 позволяет отобразить большие схемы, графики на одном листе, что полезно 
при проведении собраний, переговоров и обсуждений.

Большой объем печати определяется ресурсом расходных материалом, то есть количеством страниц которые можно напечатать используя один комплект 
чернил или тонера.

Совместимость и подключение определяется типами интерфейсов для подключения, совместимость драйверов с операционными 
системами (далее OC), а также сетевыми возможностями.
Рассмотрим в таблице 2.2 данные аспекты.

\begin{table}[ht]
    \caption{Совместимость и подключение}
    \label{table:func:printersConnectionProperty}
    \begin{tabular}{| >{\raggedright}m{0.475\textwidth}
                    | >{\raggedright\arraybackslash}m{0.475\textwidth}|}
        \hline
        \centering Аспекты подключения и совместимости & \centering\arraybackslash Описание \\

        \hline
        Типы интерфейсов для подключения &
        USB, Ethernet, Wi-Fi, Bluetooth.
        \\
        \hline
        Совместимость драйверов с ОС &
        Windows, Linux, MacOS
        \\
        \hline
        Сетевые возможности &
        Поддержка проводных и беспроводных сетей
        \\
        
        \hline
    \end{tabular}
\end{table}

Экологичность принтера можно определить его энергоэффективностью, а также наличием экологических сертификатов. 
Энергоэффективность определяется за счет мощности потребляемой принтером и его способностью переходить в спящий режим. 
Наличие сертификатов, таких как ENERGY STAR или других экологических сертификатов, свидетельствует о соответствии принтера 
определенным стандартам по энергоэффективности и экологической безопасности.

Для выбора существующих принтеров также воспользуемся сайтом-агрегатором Onliner.by. Для этого следует определиться с характеристиками
принтера в соответствии с требованиями приведенными выше.

\begin{table}[ht]
    \caption{Модели принтеров}
    \label{table:func:printersList}
    \begin{tabular}{| >{\raggedright}m{0.225\textwidth}
                    | >{\raggedright\arraybackslash}m{0.225\textwidth}
                    | >{\raggedright\arraybackslash}m{0.225\textwidth}
                    | >{\raggedright\arraybackslash}m{0.225\textwidth}|}
        \hline
        \centering Модель
        & \centering\arraybackslash Качество печати/размер
        & \centering\arraybackslash Совместимость 
        & \centering\arraybackslash Подключения \\

        \hline
        h & h & h & h
        \\
        
        \hline
    \end{tabular}
\end{table}

\subsection{Выбор модели NVR и камер видеонаблюдения}

Видео камеры в организации выполняют две важных функции, а именно:
\begin{itemize}
    \item Безопасность и контроль доступа;
    \item Мониторинг рабочих процессов;
\end{itemize}

Можно предположить, что камеры видеонаблюдения в организации могут использоваться также и в целях исследования, например, для 
поиска решений автоматизации рабочих процессов посредством анализа видеопотока.

К камерам можно выделить следующие требования для того чтобы функции камер выполнялись как требуется:
\begin{itemize}
    \item Разрешение и качество изображения;
    \item Угол обзора и область покрытия;
    \item ИК-подсветка;
    \item Наличие PoE
\end{itemize}

Разрешение и качество изображения позволяет получить четкие изображения, что может являться ключевым фактором в обеспечении безопасности. 
Угол обзора позволяет захватить большую площадь для наблюдения и соответственно сократить количество камер на одно помещение.
ИК-подсветка позволяет наблюдать в условиях плохой освещенности, например, ночью, когда никто не работает. Наличие PoE упрощает их
монтаж, так как не требуется отдельно проводить кабельные сети для питания камер.

Для мониторинга и управления камерами выбрано использование NVR. Данное устройство позволяет хранить записи с видеокамер и получать
доступ к этим данным по сети.
Для выбора NVR следует обратить внимание на следующие характеристики:

\begin{itemize}
    \item Количество каналов;
    \item Разрешение записи;
    \item Ёмкость хранения;
    \item Сетевые возможности;
    \item Поддержка PoE;
\end{itemize}

NVR может поддерживать определенное количество камер видеонаблюдения, 
что называется "количество каналов". Это определяет, сколько камер можно подключить и 
записывать с помощью данного NVR. Для подключения к сети NVR может иметь порты Ethernet, а также 
поддержку различных сетевых протоколов для удаленного доступа и управления. Поддержка PoE, требуется 
из-за требований предъявленным камерам. 

!!! Выбор конкретных устройств

\subsection{Выбор устройства для защиты от перепадов напряжения}


\subsection{Выбор модели коммутатора}

На основе технического задания следует выбрать коммутатор коммерческого уровня. Прежде всего следует выбрать производителя
сетевого оборудования, так как заказчик оставил для исполнителя выбор.
Существуют следующие популярные производители сетевого оборудования у которых имеются коммутаторы коммерческого уровня:

\begin{itemize}
    \item Cisco;
    \item HPE;
    \item Dell;
    \item D-Link;
    \item Juniper;
\end{itemize}

В таблице 2.4 приведены следующие ключевые факторы каждого производителя. Данные факторы не затрагивают характеристики конкретного оборудования,
а рассматривают производителя со стороны его популярности, качества производимого оборудования и т.д. Сравнения были взяты с популярного зарубежного
интернет-издания TechRadar.com.

\begin{table}[ht]
    \caption{Характеристики производителя}
    \label{table:func:manufacturerList:1}
    \begin{tabular}{| >{\raggedright}m{0.100\textwidth}
                    | >{\raggedright\arraybackslash}m{0.190\textwidth}
                    | >{\raggedright\arraybackslash}m{0.190\textwidth}
                    | >{\raggedright\arraybackslash}m{0.190\textwidth}
                    | >{\raggedright\arraybackslash}m{0.190\textwidth}|}
        \hline
        \centering Модель
        & \centering\arraybackslash Присутствие на рынке
        & \centering\arraybackslash Поддержка 
        & \centering\arraybackslash Инновации
        & \centering\arraybackslash Доступность \\

        \hline
        Cisco &
        Широкий спектр высококачественного сетевого оборудования. Является лидером на рынке. & 
        Имеет услуги поддержки, предлагает обучающие программы и курсы & 
        Постоянно внедряет новые технологии и стандарты &
        Дорогостоящее оборудование
        \\
        \hline
        HPE &
        Предлагает обширную линейку как сетевого оборудования так и других технологических решений. Популярен на рынке 
        оборудования коммерческого уровня &
        Хорошая поддержка, которая отвечает различным потребностям бизнеса & 
        Внедряет облачные технологии  и технологии программно-определяемых сетей (software-defined networks) &
        Конкурентная цена оправданная предлагаемыми возможностями.
        \\
        \hline
    \end{tabular}
\end{table}

\begin{table}[ht]
    \caption{Характеристики производителя}
    \label{table:func:manufacturerLit:2}
    \begin{tabular}{| >{\raggedright}m{0.100\textwidth}
                    | >{\raggedright\arraybackslash}m{0.190\textwidth}
                    | >{\raggedright\arraybackslash}m{0.190\textwidth}
                    | >{\raggedright\arraybackslash}m{0.190\textwidth}
                    | >{\raggedright\arraybackslash}m{0.190\textwidth}|}
        \hline
        \centering Модель
        & \centering\arraybackslash Присутствие на рынке
        & \centering\arraybackslash Поддержка 
        & \centering\arraybackslash Инновации
        & \centering\arraybackslash Доступность \\

        \hline
        D-Link & 
        Предлагает сетевое оборудование для домов и организаций малого и среднего размера & 
        Поддержка имеется, но мен
        \hline
        D-Link & 
        Предлагает сетевое оборудование для домов и организаций малого и среднего размера & 
        Поддержка имеется, но менее расширенная по сравнению с другими производителями & 
        Предлагает простые решения которые соответствуют современным технологическим решениям & 
        Низкая цена, подходящая для домов, малых и средних бизнесов.
        \\ее расширенная по сравнению с другими производителями & 
        Предлагает простые решения которые соответствуют современным технологическим решениям & 
        Низкая цена, подходящая для домов, малых и средних бизнесов.
        \\
        \hline
        Dell & 
        В основном производство нацелено на сервера и хранилища, но также занимает часть рынка по продаже сетевого оборудования включая коммутаторы & 
        Предлагает хорошую поддержку для своих продуктов & 
        Технологии ориентируются на программно-определяемые сети & 
        Конкурентная цена оправданная предлагаемыми возможностями.
        \\
        \hline
        Juniper & 
        Предлагает высокопроизводительное сетевое оборудование с фокусом на безопасность и надежность & 
        Качественная поддержка особенно для сложных сетевых решений & 
        Ориентируется на инновации в области высокопроизводительных решений, а также решений в области безопасности автоматизации и облачных решений & 
        Дорогостоящее оборудование
        \\
        \hline
    \end{tabular}
\end{table}

Исходя из вышеперечисленных факторов производитель Cisco является лучшим выбором за счет того, что имеет качественное оборудование, постоянно внедряет
новые технологии и является достаточно популярным производителем на рынке сетевого оборудования, 
и что не мало важно за счет своих обучающих программ имеется множество специалистов, которые могут обслуживать данное оборудование. 
Поэтому выбор сетевого коммутаторов от компании Cisco будет гарантировать НИО АСУ надежную, высокопроизводительную и обслуживаемую сеть.

Далее следует определиться с требованиями к функционалу коммутатора 
в рамках научной исследовательской организации:

\begin{itemize}
    \item Стекирование;
    \item Поддержка gigabit портов;
    \item PoE;    
    \item Поддержка VLAN;    
    \item Возможность администрирования;    
    \item Безопасность;
\end{itemize}

Первое что можно выделить это поддержка стекирования, это важный фактор в рамках НИО, потому что следует организовать 50 стационарных подключений
а также к сети будут подключены точки доступа, NVR и маршрутизаторы, ко всему этому должно иметься некоторое количество портов для возможности
организации новых подключений без затрагивания сетевого оборудования, исходя из этого следует рассчитывать на еще 2 точки доступа и 1 подключение к NVR итого 53 подключения к коммутатору.
Порты должны поддерживать как минимум Gigabit Ethernet, для поддержки DOCSIS 3.1 модема требуется поддержка 10 Gigabit Ethernet, учитывая что к сети будет подключено более 50 устройств. PoE требуется для подключения точек доступа. 
Поддержка VLAN требуется для изоляции трафика между устройствами, так как организации состоит из 6 отделов.
Возможность администрирования требуется для большего контроля сетевого оборудования в рамках организации.
Также коммутатор должен обеспечить как базовый функционал безопасности, например фильтрация MAC-адресов так и расширенный, что буде считаться большим плюсом в сторону
выбора коммутатора.

Официальный сайт Cisco предоставляет удобный веб-интерфейс для подбора оборудования и его сравнения, в таблице xx приведено сравнение следующих коммутаторов 
выбранных через функционал подбора оборудования.

\begin{table}[ht]
    \caption{Характеристики производителя}
    \label{table:func:switchList}
    \begin{tabular}{| >{\raggedright}m{0.100\textwidth}
                    | >{\raggedright\arraybackslash}m{0.200\textwidth}
                    | >{\raggedright\arraybackslash}m{0.250\textwidth}
                    | >{\raggedright\arraybackslash}m{0.100\textwidth}
                    | >{\raggedright\arraybackslash}m{0.200\textwidth}|}
        \hline
        \centering Модель
        & \centering\arraybackslash Кол-во портов
        & \centering\arraybackslash Поддержка Gigabit портов
        & \centering\arraybackslash Комм. способность
        & \centering\arraybackslash Безопасность \\

        \hline
        Catalyst 9200L &
        24-48 &
        10G/1G, 10/100/1000BASE-T copper, mGig &
        56-160 Гбит/с &
        AES-128/MACsec-128, SSH, TLS, IPsec, IGMP snooping, MPLS, NetFlow
        \\

        \hline
        Catalyst 9200CX &
        8-12 (1G), 2 x (10G SFP+) &
        10G SFP+, 1G copper &
        60-120 Гбит/с &
        SSH, TLS, IPsec, IGMP snooping, MPLS, NetFlow
        \\

        \hline
        Catalyst 9200 &
        24-48 &
        10G/1G, 10/100/1000BASE-T copper, mGig &
        128-400 Гбит/с &
        AES-128/MACsec-128, SSH, TLS, IPsec, IGMP snooping, MPLS, NetFlow
        \\

        \hline
        Catalyst 1000 &
        8, 16, 24, 48 &
        10G SFP+, 1G SFP &
        20-176 Гбит/с &
        SSH, Kerberos, SNMP v3, IGMP snooping, MPLS, NetFlow
        \\

        \hline
    \end{tabular}
\end{table}


\begin{table}[ht]
    \caption{Характеристики производителя}
    \label{table:func:switchList}
    \begin{tabular}{| >{\raggedright}m{0.100\textwidth}
                    | >{\raggedright\arraybackslash}m{0.200\textwidth}
                    | >{\raggedright\arraybackslash}m{0.250\textwidth}
                    | >{\raggedright\arraybackslash}m{0.100\textwidth}
                    | >{\raggedright\arraybackslash}m{0.200\textwidth}|}
        \hline
        \centering Модель
        & \centering\arraybackslash Кол-во портов
        & \centering\arraybackslash Поддержка Gigabit портов
        & \centering\arraybackslash Комм. способность
        & \centering\arraybackslash Безопасность \\

        \hline
        Catalyst 9300X &
        12-24 (multi-rate 1/2.5/5/10/25G SFP28), 24-48 (1/2.5/5/10G Multigigabit) &
        100G, 40G, 25G, 10G, 1G fiber &
        1-2 Тбит/с &
        AES-256/MACsec-256, SSH, TLS, IPsec, IGMP snooping, MPLS, NetFlow, Аналитика зашифрованного трафика
        \\

        \hline
        Catalyst 9300LM &
        24-48 &
        40G, 25G, 10G, 1G fiber Multigigabit, 10G/5G/2.5G/1G, 10/100/1000BASE-T copper &
        56-472 Гбит/с &
        AES-256/MACsec-256, SSH, TLS, IPsec, IGMP snooping, MPLS, NetFlow, Аналитика зашифрованного трафика
        \\

        \hline
        Catalyst 9300 &
        24-48 &
        40G, 25G, 10G, 1G fiber, Multigigabit, 10G/5G/2.5G/1G, 10/100/1000BASE-T copper &
        208-640 Гбит/с &
        AES-256/MACsec-256, SSH, TLS, IPsec, IGMP snooping, MPLS, NetFlow, Аналитика зашифрованного трафика
        \\

        \hline
        Catalyst 9300L &
        24-48 &
        40G, 25G, 10G, 1G fiber Multigigabit, 10G/5G/2.5G/1G, 10/100/1000BASE-T copper &
        56-472 Гбит/с &
        AES-256/MACsec-256, SSH, TLS, IPsec, IGMP snooping, MPLS, NetFlow, Аналитика зашифрованного трафика
        \\
        \hline
    \end{tabular}
\end{table}

С точки зрения безопасности наиболее выделяется модели коммутаторов серии Catalyst 9300*. С точки зрения поддержки Gigabit портов коммутаторы должны поддерживать 10G интернет,
так как технология DOCSIS 3.1 (рассмотрен в обзоре литературы). 

\subsection{Выбор модели точки доступа}

Выбор производителя будет основан на выборе из пункта 2.6.
Далее следует определится с количеством точек доступа которые могут покрыть этаж прямоугольного здания
с площадью 450 м.кв. и соотношением сторон 1 к 2-ум. Для данного типа помещений рекомендуются точки доступа с антенной типа ...
Для расчета покрытия рассчитаем ...

Требования к точке доступа внутри организации можно выделить следующие:
\begin{itemize}
    \item Небольшое количество подключений;
    \item Хорошее покрытие;
    \item Безопасность;
    \item Подключение через PoE;
\end{itemize}

Выбор...

\subsection{Внешняя адресация}
\subsection{Внутрення IPv4 адресация}
\subsection{Внутрення IPv6 адресация}
\subsubsection{Конфигурация DHCP и DHCPv6}
\subsection{Конфигурация сетевого оборудования}
\subsubsection{Конфигурация коммутаторов}
\subsubsection{Конфигурация маршрутизатора}
\subsubsection{Конфигурация VLAN}
\subsubsection{Конфигурация агрегации каналов}
\subsubsection{Конфигурация беспроводных точек доступа}
\subsubsection{Конфигурация cетевого экрана}